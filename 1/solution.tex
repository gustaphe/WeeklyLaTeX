\documentclass[a4paper]{article}
\usepackage[margin=2.5cm]{geometry}
\usepackage{xcolor}
\usepackage{fontspec}
\usepackage{siunitx}
\usepackage{booktabs}
\usepackage{graphicx}
\usepackage{microtype}

\setmainfont{DejaVu Serif}
\setsansfont{DejaVu Sans}

\definecolor{emphasis}{rgb}{0.4,0.1,0.1}
%\definecolor{emphasis}{rgb}{0.7,0.1,0.1} % so much fun to be had with theming
%\definecolor{emphasis}{rgb}{0.1,0.5,0.4}

\sisetup{table-parse-only,table-number-alignment=right}

\pagestyle{empty}

\newenvironment{instructions}[1]{\noindent\textsf{\textcolor{emphasis}{#1:}}}{\vspace{1ex}}

\setlength\tabcolsep{0.5ex}

\setcounter{page}{9}

\begin{document}
\noindent\begin{minipage}[t]{0.1\textwidth} % In a real case I would probably use fancyheader or something, but that's boring
        \vspace{0pt}
        \rotatebox{90}{\parbox{0.9\textheight}{\raggedleft
                        {\fontsize{48pt}{0pt} \textsf{\textcolor{emphasis}{Carnaroli Risotto}}}\\
                        {\large on Heirloom Tomato \& Basil Salad}\\[0.5ex]
                        \arabic{page}\hspace{1em}\rule[0.5ex]{0.8\textheight}{0.05ex}
        }}
\end{minipage}\hfill
\begin{minipage}[t][0.9\textheight]{0.8\textwidth}
        \vspace{0pt}
        \noindent\textit{\textcolor{emphasis}{
                        Carnaroli rice is native to the Novara and Vercelli
                        regions of northern Italy.  The grains are slightly
                        larger than the more common arborio rice, which means
                        they absorb more liquid, resulting in a slightly
                        creamier risotto.
                        \textbf{Serves 6}
        }}

        \vspace{1cm}
        \begin{minipage}[t]{0.5\textwidth}
                \vspace{0pt}\raggedright
                \begin{tabular}{Sp{5cm}}
        & \textsf{\textcolor{emphasis}{Tomato and Basil Salad}} \\
                        6 & large heirloom tomatoes (different colours if possible) \\
                        1.42 & \si{dl} basil leaves \\ % I took the liberty of converting to real units
                        53.27 & \si{ml} balsamic vinegar from Modena, Italy, or Venturi-Schulze (optional) \\
                \end{tabular}
        \end{minipage}\hfill
        \begin{minipage}[t]{0.5\textwidth}
                \vspace{0pt}\raggedleft
                \begin{tabular}{Sp{5cm}}
        & \textsf{\textcolor{emphasis}{Carnaroli Risotto}} \\
                        1.137 & \si{l} water \\
                        35.52 & \si{ml} extra-virgin olive oil \\
                        0.5 & medium onion, minced \\
                        1 & clove of garlic, minced \\
                        1.42 & \si{dl} carnaroli rice \\
                        0.71 & \si{dl} dry white wine \\
                        35.52 & \si{ml} butter \\
                        0.71 & \si{dl} grated Parmesan cheese \\
                \end{tabular}
        \end{minipage}
        \vspace{1cm}

        \begin{instructions}{Tomato and Basil Salad}
                Slice tomatoes \SI{6.35}{mm} thick and divide among six plates.
                Season with salt and pepper, then set aside.
        \end{instructions}

        \begin{instructions}{Carnaroli Risotto}
                Heat water in a medium pot on medium heat and keep it at a
                gentle simmer, turning down the heat if necessary.

                Heat oil in a large heavy-bottomed pot on medium-high heat. Add
                onion and garlic, then sauté, stirring, for about \SI{5}{min},
                or until onion is translucent.  Season well with salt and
                pepper. Stir in rice and cook for another \SI{4}{min}, or until
                the grains are well coated with oil. Add wine and
                \SI{2.841}{dl} of the simmering water, stirring constantly
                until the rice absorbs the liquid.

                Gradually add the remaining water, about \SI{1.4205}{dl} at a
                time, so rice is always covered in liquid. Keep stirring. (It
                should take about \SI{18}{min} for the rice to be perfectly
                cooked.  Taste it at various intervals near the end. You may
                need a little more or a little less water, depending on the age
                of the rice.) When all the liquid is absorbed and rice is al
                dente, stir in butter and Parmesan.  Stir vigorously to
                emulsify everything together. Season if necessary.

        \end{instructions}

        \begin{instructions}{To Serve}
                Divide risotto evenly on top of plated tomatoes, leaving at
                least one-third of the colourful tomatoes exposed. Place a few
                basil leaves around each plate and garnish with a drizzle of
                balsamic vinegar (optional).
        \end{instructions}

        \vfill
        {\centering\textit{\textcolor{emphasis}{Fresh Shiet}}\\}
\end{minipage}
\end{document}
